%% Adaptado de 
%% http://www.ctan.org/tex-archive/macros/latex/contrib/IEEEtran/
%% Traduzido para o congresso de IC da USP
%%*****************************************************************************
% Não modificar

\documentclass[twoside,conference,a4paper]{IEEEtran}

%******************************************************************************
% Não modificar
\usepackage{IEEEtsup} % Definições complementares e modificações.
\usepackage[utf8]{inputenc} % Disponibiliza acentos.
\usepackage[english]{babel}
%% Disponibiliza Inglês e Português do Brasil.
\usepackage{latexsym,amsfonts,amssymb} % Disponibiliza fontes adicionais.
\usepackage{theorem} 
\usepackage[cmex10]{amsmath} % Pacote matemático básico 
\usepackage{url} 
\usepackage{graphicx}
\usepackage{amsmath}
\usepackage{amssymb}
\usepackage{color}
\usepackage[pagebackref=true,breaklinks=true,letterpaper=true,colorlinks,bookmarks=false]{hyperref}
\usepackage[tight,footnotesize]{subfigure} 
\usepackage[noadjust]{cite} % Disponibiliza melhorias em citações.
\usepackage{tikz}
\newcommand*\circled[1]{\tikz[baseline=(char.base)]{
            \node[shape=circle,draw,inner sep=1pt] (char) {#1};}}
%%*****************************************************************************

\usepackage{listings}
\lstset{basicstyle=\footnotesize\ttfamily,  language=Python}
\renewcommand{\lstlistingname}{Code}% Listing -> Algorithm

\begin{document}
\renewcommand{\IEEEkeywordsname}{Palavras-chave}

%%*****************************************************************************

\urlstyle{tt}
% Indicar o nome do autor e o curso/nível (grad-mestrado-doutorado-especial)
\title{P2 - Simulador Robótico}
\author{%
 \IEEEauthorblockN{Isadora Sophia Garcia Rodopoulos\,\IEEEauthorrefmark{1}\\
                   Renato Landim Vargas\,\IEEEauthorrefmark{1}}
 \IEEEauthorblockA{\IEEEauthorrefmark{1}%
                   Ciência da Computação - Graduação - RA 158018\\
                   E-mail: ra158018@ic.unicamp.br}
 \IEEEauthorblockA{\IEEEauthorrefmark{1}%
                   Ciência da Computação - Graduação - RA 118557\\
                   E-mail: ra118557@ic.unicamp.br}
}

%%*****************************************************************************

\maketitle

%%*****************************************************************************
% Resumo do trabalho
\begin{abstract}
O trabalho se baseou em implementar assuntos teóricos desenvolvidos em sala de aula,
entre eles: explorar métodos de controle mais avançados e computar a odometria
do robô através de um modelo cinemático. Os assuntos são desenvolvidos a partir de
um ambiente com o robô p3dx, com o suporte do \textit{V-REP}.

O objetivo foi de levar o robô a explorar o cenário e a coletar informações relevantes tanto sobre o ambiente quanto a sua localização. A solução foi proposta através do mapeamento de informações com o ground truth - com o auxílio de gráficos, e utilizando o modelo cinemático com a odometria para o controle do robô. Para tornar o controle mais robusto, foi utilizado métodos como conjuntos \textbf{fuzzy}, para que o robô possa acompanhar as paredes, desviar de obstáculos, além de uma subrotina para evitar que fique preso.

O resultado obtido foi um sistema de controle mais robusto com sensores que se assemelham mais a uma situação do mundo real - isto é, sem a utilização de ground truth para se locomover.

% O resumo deve conter uma breve descrição sobre várias partes do seu trabalho que serão tratadas no decorrer do texto. Primeiramente, pode-se descrever brevemente o problema no qual você está trabalhando: Por que você está desenvolvendo este trabalho? Qual a motivação para este desenvolvimento? Por que ele é importante? O resumo deve conter também um breve descritivo da metodologia que você usou no desenvolvimento: Que problema foi tratado? Como a solução foi construída/desenvolvida? Quais as tecnologias utilizadas? Finalmente, deve falar um pouco sobre os resultados que você conseguiu: o resultado final ficou bom? Quais os seus principais diferenciais? Qual a eficiência do desenvolvimento?

\end{abstract}

\begin{IEEEkeywords}
fuzzy, odometria, sistemas inteligentes.

% Indique três palavras-chave que descrevem o trabalho

\end{IEEEkeywords}

%%*****************************************************************************

\section{Introdução}
O projeto foi baseado na literatura proposta em sala de aula e em artigos relacionados ao problema proposto, o qual o sistema fuzzy foi baseado \cite{Reinhard:1995}, por exemplo. O foco do projeto se deu à aproximação da simulação com problemas do mundo real, com a introdução de odometria e controladores mais robustos, com foco em conjuntos fuzzy.

Apesar de conjuntos fuzzy serem menos utilizados no estado da arte da robótica atual, há vários trabalhos de anos anteriores que o utilizam para a resolução de problemas. Além disso, ainda é um conceitp bastante utilizado para a programação de uma AI em videogames, por exemplo, devido a seu baixo custo computacional.

O trabalho encontra-se organizado da seguinte forma: a seção 2, descrevendo o modelo cinemático utilizado; a seção 3, descrevendo o processo realizado para a odometria; a seção 4, apresentando os sistemas de controle do robô; os resultados são apresentados na seção 5 e, finalmente, as conclusões são apresentadas na seção 6. 

\section{Modelo cinemático}

O modelo cinemático do robô utilizado foi o diferencial. Para isso, é necessário calcular a velocidade linear de cada uma das duas rodas, $V_{r}$ (roda direita) e $V_{l}$ (roda esquerda). Esses dois valores são calculados da seguinte forma:
\begin{gather*}
V_{r} = r * \phi_{r} \\
V_{l} = r * \phi_{l},
\end{gather*}
onde $r$ é o raio da roda e $\phi_{r}$ e $\phi_{l}$ são as velocidades angulares de cada roda. Essas velocidades são calculadas através das diferenças de posição angulares das rodas em um certo $\Delta t$, dividindo a variação angular pela variação de tempo.

Então, calcula-se as velocidades linear e angular do robô da seguinte forma:
\begin{gather*}
V = \frac{V_{r} + V_{l}}{2} \\
\omega = \frac{V_{r} - V_{l}}{2 * L},
\end{gather*}
onde L é metade da distância entre as rodas. A partir dessas duas velocidades, calcula-se a variação de distância e ângulo para uma certa variação de tempo. 
\begin{gather*}
\Delta s = V * \Delta t \\
\Delta \theta = \omega * \Delta t
\end{gather*}
Com essas variações, pode-se somar à posição atual para calcular uma posição nova.

\subsection{Sistema de referência}

Como o robô calcula apenas variações de posição e ângulo, ele não sabe sua posição exata no mundo, mas apenas no seu sistema de referência local. Para converter essa posição local para uma posição no mundo, foi implementado um conversor entre modelos de referências. Esse conversor cria e salva as matrizes de rotação e translação para converter a qualquer momento. Porém, para que isso seja possível, é necessário inserir a posição inicial do robô no mundo.

\section{Odometria}

Como não era permitido usar o ground truth do robô e os resultados do modelo cinemático básico são insatisfatórios, foi necessário calcular a sua odometria para estimar uma variação de pose mais precisa. Para um certo frame $t$, pode-se calcular essa variação por:
\begin{gather*}
\Delta \xi_{t} =
\begin{bmatrix}
\Delta s * \cos(\theta_{t-1} + \frac{\Delta \theta_{t}}{2}) \\
\Delta s * \sin(\theta_{t-1} + \frac{\Delta \theta_{t}}{2}) \\
\theta_{t}
\end{bmatrix}
\end{gather*}
Somando essa variação a cada frame, obtemos um valor de pose mais próximo da realidade comparado ao anterior.

\subsection{Correção}

Apesar de obter-se resultados melhores com a odometria, a variação de ângulo ainda é muito imprecisa e propaga muitos erros. Para um valor mais preciso, utilizou-se um giroscópio. O giroscópio originalmente retorna apenas variações de ângulo, mas como não pode-se garantir que será possível ler  o giroscópio sempre que dados novos aparecerem, ele foi modificado para somar essas variações e sempre retornar a sua estimativa de posição angular. As equações de odometria não mudam, mas apenas utilizam o novo valor de $\Delta \theta$.

\section{Sistema de controle}

O sistema de controle foi dividido em dois comportamentos distintos: o Wall Follow, que permite que o robô siga a parede ao mesmo tempo que procura desviar de obstáculos; e o Unstuck, o qual verifica se o robô está preso e, caso afirmativo, o retira desse estado.

\subsection{Wall Follow}

O Wall Follow foi baseado em conjuntos Fuzzy da literatura \cite{Reinhard:1995}. Basicamente, foram utilizadas duas regras: \textbf{directional} e \textbf{speed}, os quais regulam a direção e a velocidade do robô. Os inputs e regras de \textit{defuzzification} utilizados estão ilustrados nas figuras abaixo.

\begin{figure}[ht!]
  \centering
  \includegraphics[width=1\hsize]{figuras/per_angle.png}
  \caption{Conjunto Fuzzy do ângulo em que o robô detectou o obstáculo, ie. uma parede. Fonte \cite{Reinhard:1995}}
  \label{fig:fig1}
\end{figure}

\begin{figure}[ht!]
  \centering
  \includegraphics[width=1\hsize]{figuras/per.png}
  \caption{Conjunto Fuzzy da intensidade da percepção que o robô obteve do obstáculo. Fonte \cite{Reinhard:1995}}
  \label{fig:fig2}
\end{figure}

\begin{figure}[ht!]
  \centering
  \includegraphics[width=1\hsize]{figuras/per_change.png}
  \caption{Conjunto Fuzzy da diferença da intensidade de percepção em relação à iteração anterior. Fonte \cite{Reinhard:1995}}
  \label{fig:fig3}
\end{figure}

\begin{figure}[ht!]
  \centering
  \includegraphics[width=1\hsize]{figuras/steer.png}
  \caption{Conjunto Fuzzy para que o robô possa mudar a direção. Fonte \cite{Reinhard:1995}}
  \label{fig:fig4}
\end{figure}

\begin{figure}[ht!]
  \centering
  \includegraphics[width=1\hsize]{figuras/acc.png}
  \caption{Conjunto Fuzzy de valores para aceleração. Fonte \cite{Reinhard:1995}}
  \label{fig:fig5}
\end{figure}

Para facilitar a criação de regras e de conjuntos Fuzzy, foi utilizada a biblioteca \texttt{skfuzzy}, em python. Além disso, como os parâmetros dos inputs não eram, necessariamente, adequados para o nosso robô do experimento (devido a fatores como o peso do robô, por exemplo), foram utilizado valores de escala para a velocidade angular $v_{a} = .45$ e para a velocidade linear $v_{l} = 2$. 

Portanto, o algoritmo se baseia na seguinte iteração: \\

  \subsubsection{Encontra valor da magnitude e ângulo do obstáculo mais próximo}
    Basta encontrar o valor da menor leitura dentre os sensores, e o seu respectivo ângulo. 

    O valor é escalado para que quanto mais perto o objeto, mais próximo de 1 - ou seja, $p = 1-p'$, com $p'$ sendo a leitura mínima do sensor. O valor do ângulo é retornado em graus (dentre $-170^{\circ}$ a $170^{\circ}$). \\

  \subsubsection{Reflete os valores em nossos conjuntos Fuzzy, obtendo velocidade angular e a aceleração da velocidade}
    Para cada uma das duas regras, \textbf{directional} e \textbf{speed}, bastou-se passar os parâmetros obtidos e obter o output da \textit{defuzzification}. Para obter o valor de \texttt{perception\_change}, basta obter a diferença de percepção em relação a última iteração. 

    O código abaixo ilustra os processos realizados. \newpage

    \begin{lstlisting}[caption={Aplicação dos conjuntos fuzzy}]
angle, magnitude = self.get_perception_values()

perception_change = 
      clamp(abs(magnitude-last_magnitude))

# get ANGULAR speed for robot
directional.input['perception_angle'] = angle
directional.input['general_perception'] = magnitude
directional.compute()
v_angular = math.radians(directional.output['steer'])

# get LINEAR speed for robot
speed.input['perception_change'] = perception_change
speed.input['general_perception'] = magnitude
speed.compute()
acc = math.radians(speed.output['acceleration'])
    \end{lstlisting}

  \subsubsection{Verifica se o robô está preso}
    Após obter os valores apropriados para a atualização do robô, é verificado se o robô está em uma situação que está preso ou não. Mais detalhes sobre esse algoritmo encontram-se na próxima sessão.

  \subsubsection{Atualiza os valores}
    Finalmente, o robô atualiza os valores para a API em python, \texttt{robot.py}, a qual faz conexão com o \textit{V-REP}.

\subsection{Unstuck}
    O robô possui, ainda, uma subrotina de estado utilizada caso ele esteja preso. Basicamente, ela mantém há quanto tempo o robô está na mesma posição e com a mesma percepção dos sensores. Caso ele esteja parado há mais de $S = 5$ iterações, ele inicia o estado de \textbf{unstuck}.

    Nesse estado, sua velocidade será invertida. Entretanto, as funções aplicadas ao conjunto fuzzy ainda permanecem funcionando, permitindo mais mobilidade para o robô se retirar. Após $S$ iterações, ele sai do modo que está preso e volta a seu estado anterior.

    Alguns \textit{corner cases} foram considerados para a construção do comportamento: como o caso em que o robô fica preso enquanto está no estado \textbf{unstuck}, etc. 

    De forma a verificar se estava tudo funcionando corretamente, foram aplicados recursos de \textbf{debug} para examinar os \textit{edge} e \textit{corner cases} - como logs na tela, por exemplo.


\begin{table}[ht]
  \renewcommand{\arraystretch}{1.3}
  \centering
   \caption{Regra \textbf{directional}, que define a direção do robô. Fonte \cite{Reinhard:1995}}
   \label{tab:tab1}
   \begin{tabular}{lccccc}\hline
      $p$       & \textbf{VL} & \textbf{L}  & \textbf{M}  & \textbf{H}  & \textbf{VH} \\
      $\alpha$  &    &    &    &    &    \\ \hline
       \textbf{RB}  & HR & HR & R  & R  & C  \\ \hline
       \textbf{RF}  & C  & L  & L  & HL & HL \\ \hline
       \textbf{LF}  & C  & R  & R  & HR & HR \\ \hline
       \textbf{LB}  & HL & HL & L  & L  & C  \\ \hline
   \end{tabular}
\end{table}

\begin{table}[ht]
  \renewcommand{\arraystretch}{1.3}
  \centering
   \caption{Regra \textbf{speed}, que define a aceleração do robô. Fonte \cite{Reinhard:1995}}
   \label{tab:tab1}
   \begin{tabular}{lccccc}\hline
      $p$   & \textbf{VL}    & \textbf{L}    & \textbf{ME} & -- \\
      $p*$  & or \textbf{VH} & or \textbf{H} &    &    \\ \hline
      \textbf{ZE}  & ZE    & P    & P  & -- \\ \hline
      \textbf{L}   & EB    & B    & Z  & -- \\ \hline
      \textbf{H}   & --    & --   & -- & EB \\ \hline
   \end{tabular}
\end{table}

\section{Resultados e Discussão}

Para análise dos resultados, foi extraída uma nuvem de pontos representando as posições calculada (em verde) e ground true (em azul) do robô, as posições com obstáculos detectados pelos sonares (em preto), e também as posições onde o robô foi detectado como preso (em vermelho). Um exemplo do código rodando pode ser visto na Figura~\ref{fig:plot1}.

\begin{figure}[ht]
\centering
\includegraphics[width=1\hsize]{figuras/figure_3.png}
\caption{Um exemplo de plot gerado. Os círculos amarelos representam momentos a serem discutidos.}
\label{fig:plot1}
\end{figure}

Nesse exemplo, o robô começa sua posição em \circled{1}. Entre esse momento, e o momento \circled{2}, as linhas azul e verdes são indicerníveis por terem um valor muito próximo. Porém nota-se que em \circled{2}, o robô fica preso por um tempo tentando andar através de um obstáculo. Nesse instante, ocorre derrapagem e a posição estimada do robô é comprometida. Isso é notável no caminho até o momento \circled{3}, onde as linhas já estão relativamente separadas. Porém, nota-se que a orientação estimada continua alinhada com a orientação real, não sendo afetada por erros de derrapagem por ser calculada através de um giroscópio.

Nesse exemplo também é possível ver que o robô executando $Wall Follow$ de fato segue paredes, mas em vários momentos esbarra em paredes e fica preso. Nesses momentos, o robô entra em modo $Unstuck$ e eventualmente consegue sair. Porém, devido à tendência de ficar preso, as estimativas de pose do robô tendem a piorar nesses momentos.

\section{Conclusões}

Em geral, o trabalho foi consistente em relação à aplicação de conceitos aprendidos em sala de aula, além da exploração e utilização de soluções alternativas. Os resultados foram relativamente satisfatórios, com uma odometria consistente e um sistema de controle razoavelmente esperto. Entretanto, ainda há bastante espaço para explorar na questão da inteligência e aprendizado do robô, que continua limitado a alguns parâmetros, sem muita informação a respeito do ambiente. 

No desenvolvimento de futuras aplicações, seria considerada a possibilidade de estruturar com antecedência alguns parâmetros em relação à percepção do robô - talvez estruturndo-as melhor com outros recursos de sensibilidade oferecida pelo \textit{V-REP}, ou testar edge cases que sejam mais tangentes ao mundo real.

%******************************************************************************
% Referências - Definidas no arquivo relatorio.bib

\bibliographystyle{IEEEtran}

\bibliography{relatorio}

%******************************************************************************

\end{document}
